%----------------------------------------------------------------------------------------
%	PACKAGES AND OTHER DOCUMENT CONFIGURATIONS
%----------------------------------------------------------------------------------------

\documentclass{resume} % Use the custom resume.cls style

\usepackage[left=0.75in,top=0.6in,right=0.75in,bottom=0.6in]{geometry} % Document margins

\name{Ashley Penney} % Your name
\address{Boston, MA} % Your address
\address{(978)~$\cdot$~489~$\cdot$~4019 \\ apenney@gmail.com} % Your phone number and email

\begin{document}

%----------------------------------------------------------------------------------------
%	WORK EXPERIENCE SECTION
%----------------------------------------------------------------------------------------

\begin{rSection}{Experience}

\begin{rSubsection}{Maxwell Health}{July 2016 -- Present}{Director of Infrastructure}{Boston, MA}
\item[] At Maxwell Health, I turned around an operations team in crisis.  After hiring
to bring the team up to full capacity, we built up the infrastructure to turn a six week
lead time for servers into an instant self service model using Kubernetes.

I split my time between hands on work, such as core infrastructure building,
mentoring team members, working with engineering teams to transform their
build and deployment pipelines, as well as taking part in SOC2/HIPAA compliance
work, product strategy planning, and implementing a Production Readiness Review
process.

\end{rSubsection}

\begin{rSubsection}{nToggle}{January 2015 -- July 2016}{Techops Manager}{Boston, MA}
\item[] At nToggle, I rejoined several of my Jumptap coworkers to help launch
an adtech startup that optimizes and reduces Real Time Bidding traffic for
our customers.

I designed, and built, our loadbalancer-less (anycast backed) datacenter design
to deliver our high performance, low latency, platform.  Capable of
processing millions of QPS in real time, our platform has helped a variety of
bidders to significantly reduce their traffic, raise their bid rates, and
improve the delivery of campaigns to drive significant revenue upside.  

I split my time between hands-on engineering work and managing the operations
team.  This included hiring budgets, cost models and board presentations around
CAPEX and OPEX spend, as well as vendor relationships and pre-sales engineering
work with our customers to accurately model their current infrastructure costs
and demonstrate the significant RoI they can achieve through nToggle.
\end{rSubsection}

\begin{rSubsection}{Metacloud}{September 2014 -- January 2015}{Senior Software Engineer}{Pasadena, CA}
\item[] I joined the engineering team at Metacloud to transform the aging QA environment.  This was previously managed by hand and I spent several months building out Chef automation to rebuild and scale out the testing infrastructure.  This eventually led to a migration that reduced testing cycles from full days to less than an hour.  Metacloud was acquired by Cisco during this time.
\end{rSubsection}

\begin{rSubsection}{Puppet Labs}{July 2013 -- September 2014}{Senior Software Engineer}{Portland, OR}
\item[] As the first team hire, I was able to initiate and manage key projects that allowed Puppetlabs to build a developer community around the previously ailing, but critical, set of public modules provided by Puppetlabs.

These projects included our "Supported Modules" initiative, the first commercial, vendor backed set of  automation modules.  I redesigned, refactored, and rewrote several of these critical modules.  I also designed our testing strategy and worked with the team to build thousands of unit and acceptance tests across the module base, allowing us to validate them against 27 supported Puppet Enterprise Platforms.

Other work included refactoring core Puppet code, writing several popular guides around building modules, overhauling documentation, and launching weekly triage meetings to decrease the backlog of unmerged PRs.  This lead to over 2000 PR merges.
\end{rSubsection}

%------------------------------------------------

\begin{rSubsection}{Jumptap}{March 2013 - June 2013}{Senior Systems Administrator}{Boston, MA}
\item[] I joined Jumptap to help the operations team adopt devops practices and to improve their ability to support a large clustered trading and advertising platform.  I restructured the daily workflow within the operations team, using Kanban, to help transform the internal communications and tracking of work in progress.  We were able to leverage these process changes to rebuild large portions of the Puppet automation, replace Perforce with Github Enterprise, and launch a custom deployment system.

\end{rSubsection}

%------------------------------------------------

\begin{rSubsection}{EdX}{October 2012 - March 2013}{Devops Engineer}{Cambridge, MA}
\item[] Bringing my prior Puppet experience to EdX I was able to help the team turn around a difficult automation situation.  I started by rebuilding the Puppet infrastructure from scratch, rewriting all the existing code, and then moved to building Fabric tasks to automate our cloud provisioning and code deployment.

Alongside this work I started weekly reports to the leadership team, and adopted  Kanban and Agile techniques within the team in order to help us identify and overcome some bottlenecks in our infrastructure delivery pipeline.  Lastly, I worked closely with development to scope, build, and test some enhancements to the EdX platform to allow for third party integrations and sandboxing of externally submitted student code.
\end{rSubsection}

\begin{rSubsection}{SilverSky}{June 2011 - October 2012}{Team Lead, Operations}{Boston, MA}
\item[] In order to pass an FDIC FFIEC audit, I led a team of three devops engineers dedicated to fully automating all provisioning, deployment, and ongoing operations.

My team built a new multi-datacenter Puppet deployment in order to rebuild and automate all existing servers across physical and cloud environments.  We worked closely with the existing operations team to mentor junior members and spread devops practices.  This consisted of training sessions in writing Puppet modules, writing unit and acceptance tests, automated provisioning and deployment techniques, and adopting Kanban and Agile workflows.  This work led to successfully passing the audit.
\end{rSubsection}

\begin{rSubsection}{Harvard Law School}{Febuary 2008 - June 2011}{Senior Systems Administrator}{Cambridge, MA}
\item[] As part of the operations team at HLS, I introduced automation into the environment for the first time.  I built a provisioning system, and then used it to rebuild all manually maintained systems in a fully automated fashion.  This allowed us to replace 35 physical servers with 65 virtual servers, reducing delivery time of new instances from 1 week to 10 minutes.

Other work included launching the new Harvard Law website with zero downtime, and tightly integrating operations with development throughout the multi-year development of the identity management system that manages the lifecycle of student, staff, and faculty accounts and data throughout their tenure at HLS.
\end{rSubsection}

\begin{center}
Prior roles at: NTT Europe Online, MessageLabs, Tiscali, Yahoo!, Versatel, Interxion, and Speedport.\\
\textit{Please ask for an expanded resume covering those positions if you're interested in details.}
\end{center}

\end{rSection}

\end{document}
